\documentclass[12pt,fleqn]{article}
\usepackage{vkCourseML}
\usepackage{lipsum}
\usepackage{indentfirst}
\title{Машинное обучение, ФКН ВШЭ\\Домашнее задание №6}
\author{}
\date{}
\theorembodyfont{\rmfamily}
\newtheorem{esProblem}{Задача}

\begin{document}
    \maketitle
    
    \begin{esProblem}[3 балла]
        Пусть $x \in \mathbb{R}^d$, и значение каждого признака на объекте $x$ независимо генерируется из равномерного распределения: $x_i \sim U[0, 1], \, i = 1, \dots, d$. Будем считать, что объекты в выборке независимы, а $\mathbb{E}[y|x] = x^T x$. Найдите смещение константного алгоритма: $\mu(X)(x) = C = \text{const}$.    
        
    \end{esProblem}
    
    \begin{esProblem}[3 балла]
         Предположим, что объекты описываются единственным категориальным признаком, принимающим значения $x = 1, \dots, K$ с равной вероятностью, объекты независимы. Для каждой категории $x$ определено истинное целевое значение $f_x$, а наблюдаемое целевое значение для объекта x определяется как $y = f_x + \varepsilon$, $\varepsilon \sim \mathcal{N}(0, \sigma^2)$. Алгоритм обучения $\mu(X)$ запоминает среднее значение для каждой категории следующим образом: $\hat f_x = \frac 1 \ell \sum_{i=1}^\ell [x_i=x] \, y_i$, а затем для объекта $x$ выдаёт предсказание $\hat f_x$. Найдите смещение такого алгоритма.
        
    \end{esProblem}
    
    \begin{esProblem}[2 балла]
       Предположим, что мы решаем задачу бинарной классификации и что у нас есть три алгоритма~$b_1(x)$, $b_2(x)$ и $b_3(x)$,
        каждый из которых ошибается с вероятностью $p$.
        Мы строим композицию взвешенным голосованием: алгоритмам присвоены значимости $w_1$, $w_2$ и $w_3$,
        и для вынесения вердикта суммируются значимости алгоритмов, проголосовавших за каждый из классов:
        \begin{align*}
            &a_0 = \sum_{i = 1}^{3} w_i [b_i(x) = 0],\\
            &a_1 = \sum_{i = 1}^{3} w_i [b_i(x) = 1].
        \end{align*}
        Объект~$x$ относится к классу, для которого такая сумма оказалась максимальной.
        Например, если первые два алгоритма голосуют за класс $0$,
        а третий --- за класс $1$, то выбирается класс $0$, если~$w_1 + w_2 > w_3$, и класс~$1$ в противном случае.
        Какова вероятность ошибки такой композиции этих трех алгоритмов, если:
        \begin{enumerate}
            \item $w_1 = 0.2$, $w_2 = 0.3$, $w_3 = 0.2$;
            \item $w_1 = 0.2$, $w_2 = 0.5$, $w_3 = 0.2$?
        \end{enumerate}

    \end{esProblem}

\begin{esProblem}[2 балла]
	На лекции было показано, что для задачи регрессии случайный лес можно трактовать как метрический алгоритм со своеобразной функцией расстояния. Покажите, что аналогичное утверждение верно для задачи классификации, если считать, что в листьях дерева возвращаются вектора частот классов, композиция подразумевает усреднение этих векторов, и на основе этого усредненного вектора принимается решение о классе объекта.
	
\end{esProblem}
\end{document}

\documentclass[12pt,fleqn]{article}
\usepackage{vkCourseML}

\theorembodyfont{\rmfamily}
\newtheorem{esProblem}{Задача}

\title{Машинное обучение, ФКН ВШЭ\\Теоретическое домашнее задание №11}
\author{}
\date{}

\begin{document}
\maketitle

\begin{esProblem}
	Пусть целевая переменная имеет отрицательные биномиальное распределение с~фиксированным параметром~$r$:
	$$p(y \cond \theta(x)) = C_{y + r - 1}^y \theta(x)^y (1-\theta(x))^r.$$
	Запишите оптимизационную задачу поиска вектора весов модели~$w$ для соответствующей обобщенной линейной модели (для метода максимального прадоподобия), выразите значение параметра $\theta(x)$ через оптимальное значение $w^*$ и найдите матожидание ответов $y$, обусловленное натуральным параметром $\eta(x)$ через дифференцирование логарифма нормировочной константы экспоненциальной формы распределения. Совпадает ли оно с $\theta(x)$ и почему? Как будет выглядеть итоговый алгоритм прогнозирования $a(x)$?
\end{esProblem}

\begin{esProblem}
	Пусть целевая переменная имеет распределение Парето с фиксированным параметром $k$:
	$$p(y \cond \alpha(x)) = \frac{\alpha(x) k^{\alpha(x)}}{y^{\alpha(x) + 1}}.$$
	
	Запишите оптимизационную задачу поиска вектора весов модели~$w$ для соответствующей обобщенной линейной модели (для метода максимального прадоподобия), выразите значение параметра $\alpha(x)$ через оптимальное значение $w^*$ и найдите матожидание достаточной статистики $s(y)$, обусловленное натуральным параметром $\eta(x)$ через дифференцирование логарифма нормировочной константы. Совпадает ли оно с $\alpha(x)$ и почему? Какой из вариантов является лучшим кандидатом на роль алгоритма $a(x)$?
	
	\textbf{Подсказка.} Напомним, что в записи экспоненциальной формы распределения может фигурировать не $y$, а некоторая статистика $s(y)$.
\end{esProblem}
\end{document} 
